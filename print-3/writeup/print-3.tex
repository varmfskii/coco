\documentclass{article}
\usepackage{listings}
\usepackage{hyperref}
\lstdefinelanguage[Motorola6809]{Assembler}{
    morekeywords={ABX,ADCA,ADCB,ADDA,ADDB,ADDD,ANDA,ANDB,ANDCC,ASL,ASLA,ASLB,
      ASR,ASRA,ASRB,BCC,LBCC,BCS,LBCS,BEQ,LBEQ,BGE,LBGE,BGT,LBGT,BHI,LBHI,BHS,
      LBHS,BITA,BITB,BLE,LBLE,BLO,LBLO,BLS,LBLS,BLT,LBLT,BMI,LBMI,BNE,LBNE,BPL,
      LBPL,BRA,LBRA,BRN,LBRN,BSR,LBSR,BVC,LBVC,BVS,LBVS,CLR,CLRA,CLRB,CMPA,
      CMPB,CMPD,CMPS,CMPU,CMPX,CMPY,COM,COMA,COMB,CWAI,DAA,DEC,DECA,DECB,EORA,
      EORB,EXG,INC,INCA,INCB,JMP,JSR,LDA,LDB,LDD,LDS,LDU,LDX,LDY,LEAS,LERU,
      LEAX,LEAY,LSL,LSLA,LSLB,LSR,LSRA,LSRB,MUL,NEG,NEGA,NEGB,NOP,ORA,ORB,ORCC,
      PSHS,PSHU,PULS,PULU,ROL,ROLA,ROLB,ROR,RORA,RORB,RTI,RTS,SBCA,SBCB,SEX,
      STA,STB,STD,STU,STS,STX,STY,SUBA,SUBB,SUBD,SWI,SWI2,SWI3,SYNC,TFR,TST,
      TSTA,TSTB,FCB,FDB,FQB,FCC,FCN,FCS,ZMB,ZMD,ZMQ,RMB,RMD,RMQ,INCLUSEBIN,
      FILL,ORG,REORG,EQU,SET,SETDP,ALIGN,IFEQ,IFNE,IFGT,IFGE,IFLT,IFLE,IFDEF,
      IFPRAGMA,IFNDEF,ELSE,ENDC,OS9,MOD,EMOD,INCLUDE,END,ERROR,WARNING,MODULE},
    sensitive=false,
    morecomment=[l];
  }[keywords,comments,strings]

\lstdefinelanguage[Hitachi6309]{Assembler}[Motorola6809]{Assembler}{
  morekeywords={ADCD,ADCR,ADDE,ADDF,ADDW,ADDR,AIM,ANDD,ANDR,ASLD,ASRD,BAND,
    BEOR,BIAND,BIEOR,BIOR,BITMD,BOR,CLRE,CLRF,CLRD,CLRW,CMPE,CMPF,CMPW,CMPR,
    COME,COMF,COMD,COMW,DECE,DECF,DECD,DECW,DIVD,DIVQ,EIM,EORD,EORR,INCE,INCF,
    INCD,INCW,LDE,LDF,LDW,LDQ,LDBT,LDMD,LSLD,LSRD,LSRW,MULD,NEGD,OIM,ORD,ORR,
    PSHSW,PSHUW,PULSW,PULUW,ROLD,ROLW,RORD,RORW,SBCD,SBCR,SEXW,STE,STF,STW,STQ,
    STBT,SUBE,SUBF,SUBW,SUBR,TFM,TIM,TSTE,TSTF,TSTD,TSTW},
  }[keywords,comments,strings]

\title{PRINT\#-3}
\author{Theodore (Alex) Evans\\varmfskii@gmail.com}
\begin{document}
\maketitle

Complete code for this can be found at: \url{https://github.com/varmfskii/coco/tree/main/print-3/code}

Extended Color BASIC supports devices -2 to 0 with -2 being the
printer, -1 the cassette, and 0 being the basic text screen. With Disk
BASIC adds devices 1-15 as handles to open files on the disk
drive. Super Extended Color Basic on the CoCo 3 this didn't change
this other than device 0 getting a set of three modes (WIDTH 32, 40,
and 80) to support the old and new text modes. There have been various
things done to repurpose some of these devices, most notably allowing
text display on the PMODE 4 screen to allow 53, 64, or 85 column text.

I am doing something a little different. I am adding device number -3
to support a new serial channel. The method is demonstrative, and not
particularly practical, though someone may find a use for the
particular case. Some potential uses for this could be an easy way to
simultaneously use the second display provided by hardware like the
Wordpak 2+ or SuperSprite FM+ which still having easy access to the
original text screen.

There are two common ways to modify/extend the Coco's BASIC. You can
patch the 25 vectors that run through addresses \$015E to \$01A8 or,
on a machine with 64k or more RAM, directly patch BASIC. I will do a
combination of the two.

For all of the vector code.

If we have device -3:

\begin{itemize}
\item Remove one return address from the stack to inhibit BASIC's
  handling of the vector.
\item Perform our processing for device -3
\item Return
\end{itemize}

If we do not have device -3, we call the original code.

Things to do:
\begin{itemize}
\item Change device 0 to not use all of screen.
\item Validate device number:  RVEC1
\item Set print parameters: RVEC2
\item Console out: RVEC3
\end{itemize}

\section{Modify device 0}
Change device 0 to not use all of screen: Printing on device 0 uses a
routing for PRINT@ which has no hook, so for this part we are going to
modify BASIC directly. We are patching where the screen ends.

To do this first make sure the machine is in all RAM mode with the
contents of the ROMS copied to ram.

\lstinputlisting{../code/allram.asm}

Then perform the actual patches.

\lstinputlisting{../code/print0.asm}

If we don't need to change the existing device, we can add our vectors
without using all RAM mode. We also had the option of using some of
the other vectors like the PRINT vector (RVEC9) and the CLS vector
(RVEC22).

\section{Valildate device number}
Vector RVEC1 takes a value in the B register and either returns on
success or calls the error handler on a failure. In our case we want
to check for device -3, if we have it we return success otherwise we
pass control to the original routine.
exrepl st

\lstinputlisting{../code/vector1.asm}

\section{Set print parameters}
Vector RVEC2 checks what the current device is in DEVNUM and assigns
the proper values to: DEVCFW (tab zone width), DEVLCF (beginning of
last tab zone), DEVPOS (current column), and DEVWID (total number of
columns)

\lstinputlisting{../code/vector2.asm}

\section{Console out}
Vector RVEC3 takes the output character in register A and prints it to
the selected device.

\lstinputlisting{../code/vector3.asm}

\end{document}
